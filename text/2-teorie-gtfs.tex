\chapter{GTFS}
\label{2-teorie-gtfs}

General Transit Feed Specification, zkráceně GTFS, je dokument, který definuje
obecný formát pro jízdní řády veřejné dopravy a příbuzné geografické informace.
GTFS \uv{feeds} umožňují veřejným dopravním agenturám zveřejňovat svá přepravní
data a vývojářům psát aplikace, které tato data spotřebovávají interoperabilním
způsobem neboli vícesystémovým mezinárodním provozuschopným způsobem. \cite{gtfs-info}

\section{Historie GTFS}
V Portlandu ve státě Oregon v USA se společnost TriMet pokusila jako jedna z prvních 
řešit problém s plánováním tranzitní dopravy pomocí otevřených dat sdílených širokou veřejností.
V roce 2005 společnost TriMet oslovila společnost Google s dotazem, zda mají nějaké plány
na začlenění tranzitní dopravy do svých plánovačů výletů na základě veřejných údajů TriMet.
Google jim kladně odpověděl a spojily síly s implementací jednoho z prvních plánovačů výletů v Portlandu.

Jedním z prvních problémů, kterým TriMet a Google čelily, byl problém udržitelných dat
- pro zajištění kvalitních cest by plánovač cest potřeboval přepravní řád, 
trasu a údaje o zastávkách v elektronickém formátu, který by byl neustále aktuální. 
Společnost TriMet ve spolupráci se společností Google naformátovala svá přepravní 
data do snadno udržovatelného a spotřebního formátu, který lze importovat do Map Google. 
Tento formát dat přepravy se stal známým jako Specifikace zdroje Google Transit (anglicky
Google Transit Feed Specification (GTFS)). 
V roce 2005 byla tato služba plánování cesty spuštěna jako Google Transit.

Od tohoto roku se GTFS stal nejpopulárnějším datovým formátem pro přepravní služby na světě. 
Spousta agentur se rozhodla sdílet své GTFS údaje s veřejností, zatímco některé agentury 
zůstaly zdrženlivé a přístup k datům nechaly jen některým partnerům. Ke 2. prosinci 2019
uvádí OpenMobilityData 1233 poskytovatelů s veřejně přístupnými kanály GTFS,
z nichž 465 je ve Spojených státech. 

V důsledku inovací vývojářů třetích stran jsou data GTFS nyní využívána různý\-mi softwarovými aplikacemi
třetích stran k mnoha různým účelům, včetně plánování výletů, map, vytváření jízdních řádů, mobilních dat,
vizualizace, přístupnosti, analytických nástrojů pro plánování a informační systémy v reálném čase.
V roce 2010 byl název formátu GTFS změněn na General Transit Feed Specification,
aby přesně reprezentoval jeho použití v mnoha různých aplikacích mimo produkty Google. \cite{transitwiki} 
 
% doplnit něco o historii GTFS 

\section{GTFS dataset}
GTFS \uv{feed} nebo lépe jako GTFS dataset\footnote{kolekce dat, která by měla být publikována na permanentní \zk{URL} adrese}
je \zk{ZIP} soubor, který obsahuje \zk{CSV} soubory.

\zk{CSV}, \textit{Comma-separated values}, v překladu \textit{hodnoty oddělené čárkami}, je jednodu\-chý 
souborový formát určený pro výměnu tabulkových dat. Data jsou oddělována \uv{oddělovačem}.
Ačkoli podle definice by měl být formát oddělen čárkami, oddělovač může být prakticky cokoli. 
Nejčastějšími oddělovači jsou právě čárky, středníky nebo mezery. \zk{CSV} soubor se 
může upravovat v jakémkoliv textovém editoru.

\begin{figure}[H] \centering
    \includegraphics[width=250pt]{./pictures/ukazka-csv.PNG}
    \caption[Ukázka \zk{CSV} formátu ze souboru shapes.txt]{Ukázka \zk{CSV} formátu ze souboru shapes.txt}
	\label{fig:ukazka-csv}              
\end{figure}

V GTFS datasetu může být v současné době až 17 \zk{CSV} souborů v textové podobě. Slovem \uv{může} je myšleno to,
že některé \zk{CSV} soubory jsou požadované či podmíněně požadované a jiné volitelné.
Jaké \zk{CSV} soubory obsahuje dataset záleží na konkrétním dopravním systému, který
tento dataset vytváří.

Termín \uv{požadované} znamená, že daný \zk{CSV} soubor se musí nacházet v GTFS datasetu nebo dané pole
se musí nacházet v \zk{CSV} souboru a v tomto poli musí být uvedena hodnota pro každý záznam. 

%přidat referenci na podmínky?
Termín \uv{podmíněně požadované} znamená, že daný \zk{CSV} soubor nebo pole je vyžadován za určitých podmínek, 
které jsou uvedeny v popisu souboru nebo pole. Mimo tyto podmínky je tento soubor nebo pole volitelný.

Termín \uv{volitelné} znamená, že daný \zk{CSV} soubor nebo pole může být vynecháno. V případě zahrnutí 
volitelného pole mohou být některé položky prázdné řetězce, což je ekvivalentní s prázdným
polem.

V následující tabulce \ref{table:csv-soubory} jsou přehledně zobrazeny všechny \zk{CSV} soubory,
které v současnosti v GTFS datasetu mohou být.

\newcolumntype{s}{>{\centering\arraybackslash\columncolor[HTML]{CCFFCC}} m{5cm}}
\newcolumntype{v}{>{\centering\arraybackslash\columncolor[HTML]{C4FFFD}} m{5cm}}
\setlength{\arrayrulewidth}{0.3mm}
\begin{table}[h!]
\begin{center}
\begin{tabular}{ |s|v| } 
  \hline
  požadované/podmíněně požadované & volitelné \\ 
  \hline
  agency.txt & fare\_attributes.txt \\ 
  stops.txt & fare\_rules.txt \\ 
  routes.txt & shapes.txt \\
  trips.txt & frequencies.txt \\
  stop\_times.txt & transfers.txt \\
  calendar.txt & pathways.txt \\
  calendar\_dates.txt & levels.txt \\ 
  feed\_info.txt & translations.txt \\
  - & attributions.txt \\ 
  \hline      
\end{tabular}
\end{center}
\caption{Seznam \zk{CSV} souborů v GTFS datasetu}
\label{table:csv-soubory}
\end{table}

Každý \zk{CSV} soubor v GTFS datasetu má v prvním řádku názvy polí, do kterých je tento
soubor rozřazen. Jednotlivá pole mají různý datové typy, které jsou barva, kód měny, 
datum, email, enum (výčet), \zk{ID}, kód jazyka, zeměpisná délka, zeměpisná šířka,
nezáporné číslo s plovoucí desetinnou čárkou, nezáporné celé číslo, telefonní číslo,
čas, text, časové pásmo a \zk{URL} adresa.

Jedno z nejdůležitějších polí je pole s datovým typem \zk{ID}, což je hodnota jednoznačně určující každý záznam.
Právě tento datový typ umožňuje propojení jednotlivých \zk{CSV} souborů mezi sebou. \zk{ID} může být
sekvence libovolných UTF-8 znaků. Pole s datovým typem \zk{ID} se označují na konci názvu s
\uv{\_id}. Na následujícím obrázku \ref{fig:GTFS-diagram} je toto propojení zobrazeno pomocí diagramu.

%přidat referenci na podmínky?
Na obrázku \ref{fig:GTFS-diagram} je taktéž tučně zobrazeno, které pole v daném \zk{CSV} souboru
jsou požadované, podmíněně požadované nebo volitelné.  

\begin{figure}[H] \centering
    \includegraphics[width=400pt]{./pictures/GTFS-diagram.PNG}
    \caption[Diagram GTFS datasetu]{Diagram GTFS datasetu}
	\label{fig:GTFS-diagram}              
\end{figure}

Pro moji diplomovou práci byly dále důležité datové typy jako zeměpisná délka a zeměpisná šířka,
které již podle názvu obsahují zeměpisnou délku a šířku v souřadnicovém systému WGS84, barva zakódovaná 
jako šestimístné hexadecimální číslo, nezáporné číslo s plovoucí desetinnou čárkou a nezáporné celé číslo
nebo enum, což jsou předem definované konstanty.

\subsection{Soubor stops.txt}
\label{stops.txt}
Soubor \textit{stops.txt} se skládá ze 14 polí, z čehož 6 polí je požadovaných nebo podmíněně požadovaných 
a zbytek volitelných. Samotný soubor je požadovaný a měl by se nacházet v každém GTFS datasetu.

Prvním polem je vždy zpravidla \textit{stop\_id}, které je požadované a má datový typ \zk{ID}.
Tato hodnota jednoznačně určuje každou zastávku. Pro PID\_GTFS, což je GTFS dataset Pražské integrované dopravy,
se pole \textit{stop\_id} skládá z kombinace písmen a čísel v souvislosti s typem spoje.

Druhým polem je volitelné pole \textit{stop\_code} v datovém typu text, což je krátký text nebo číslo, 
které identifikuje lokaci pro řidiče. 

Třetí pole je s datovým typem text \textit{stop\_name}. Jak už název napovídá, tak obsahuje název lokace. 
Je podmíněně požadované kvůli dalšímu volitelnému devátému poli \textit{location\_type} s datovým typem enum, 
které obsahuje druhy lokace.
Toto pole je definováno pomocí čtyř konstant:
\begin{itemize} 
\item 0 nebo prázdné - zastávka nebo nástupiště
\item 1 - železniční stanice 
\item 2 - vchod/východ ze železniční stanice 
\item 3 - místo ve stanici, které neodpovídá s žádnou z ostatních konstant z \textit{location\_type} 
\item 4 - specifické místo, kde lidé mohou nastoupit/vystoupit z vozidla
\end{itemize}

Čtvrtým polem je volitelné pole \textit{stop\_desc} v textovém datovém typu obsahující popis místa, 
které poskytuje užitečné a kvalitní informace.

Pátým a šestým polem je \textit{stop\_lat} a \textit{stop\_lon} s datovým typem zeměpisná šířka a délka
obsahující přesně tyto dvě hodnoty. Tyto dvě pole jsou podmíněně požadované kvůli poli \textit{location\_type}.

Sedmým polem s datovým typem \zk{ID} je pole \textit{zone\_id}, které je pro tuto diplomovou práci obzvlášť důležité.
Je podmíněně požadované kvůli \zk{CSV} souboru \textit{fare\_rules.txt}, pokud jsou v něm poskytovány informace o jízdném.
Pokud záznam v \zk{CSV} souboru \textit{stops.txt} představuje stanici nebo vchod do stanice, je \textit{zone\_id} ignorováno.

Osmým polem je volitelný \textit{stop\_url} s \zk{URL} adresou na webovou stránku o místě záznamu.

Deváté pole je \textit{location\_type} a bylo vysvětleno společně s třetím polem.

Desátým polem je podmíněně požadované pole \textit{parent\_station} opět kvůli propojení s polem \textit{location\_type}.
Má datový typ \zk{ID} odkazující na pole \textit{stop\_id} a definuje hierarchii mezi různými místy definovanými v \textit{stops.txt}. 
a obsahuje \zk{ID} nadřazeného umístění.

Poslední čtyři pole jsou volitelné. Prvním z nich je \textit{stop\_timezone} s datovým typem časového pásma,
\textit{wheelchair\_boarding} s datovým typem enum, které označuje, zda je z daného místa možné nastupovat na invalidní vozík.
Předposlední je \textit{level\_id} s datovým typem \zk{ID} odkazující na soubor \textit{levels.txt} defi\-nující úroveň
umístění zastávky a poslední je \textit{platform\_code} s datovým typem text, 
což je identifikátor platformy pro zastávku patřící stanici.

Všechny tyto pole mají pevné pořadí a nesmí se přeházet. Na obrázku \ref{fig:stops} je ukázka \zk{CSV} souboru \textit{stops.txt}
pro PID\_GTFS dataset, kde nejsou využívány všechny volitelné pole.

\begin{figure}[H] \centering
    \includegraphics[width=400pt]{./pictures/stops.PNG}
    \caption[Ukázka \zk{CSV} souboru stops.txt pro PID\_GTFS dataset]{Ukázka \zk{CSV} souboru stops.txt pro PID\_GTFS dataset}
	\label{fig:stops}              
\end{figure} 