\chapter{Použité technologie}
\label{4-technologie}

\section{QGIS}

QGIS je profesionální geografický informační systém.
Software je zdarma ke stáhnutí a je tzv. open-source (zdrojový kód je veřejně).
Zdrojový kód je publikován ve službě GitHub (viz podkapitola \ref{section-github}).
Vývojář softwaru může být kdokoliv, avšak potvrzovat a kontrolovat změny můžou jen ověření
přispěvatelé. Software byl vyvíjen v programovacím jazyku C++ a Python (viz \ref{section-python})

\begin{figure}[H] \centering
    \includegraphics[width=64pt]{./pictures/qgis-logo.png}
    \caption[Logo systému QGIS]{Logo systému QGIS \cite{qgis}}
	\label{fig:qgis-logo}                                
\end{figure}

QGIS je oficiálním projektem nadace OSGeo (Open Source Geospatial Foundation), což je nevládní 
nezisková organizace, jejíž cílem je podporovat a prosazovat společný vývoj otevřených geoinformačních
technologií. Běží na operačních systémech Linux, Unix, Mac OSX, Windows a Android.
Rastrové a vektorové formáty podporované systémem QGIS mohou být uloženy v souborech anebo databázích.


Kromě desktop verze systému QGIS existuje i QGIS Server, který umožňuje publikovat projekty a vrstvy
QGIS jako služby \zk{WMS}, \zk{WMTS}, \zk{WFS} a \zk{WCS} kompatibilní s OGC (Open Geospatial Consortium - mezinárodní standardizační organizace
o geoprostorových datech a službách). Dále existuje webový klient, kde lze publikovat
QGIS projekty a využívat funkce symboliky a značení. Pro mobilní zařízení tu jsou
různé aplikace jako QField, Input a IntraMaps Roam, se kterými také lze používat QGIS projekty a to 
například přímo v terénu. 

%rozhraní QGIS?

QGIS má jako většina softwarů svou dokumentaci, která je sestavena pomocí nástroje Sphinx. 
Dokumentace je taktéž open-source na službě GitHub a upravovat ji může každý. Obsahuje
informace pro běžné desktop uživatele systému QGIS, pro přispěvatele dokumentace nebo pro QGIS
vývojáře. V dokumentaci si může uživatel například dočíst informace nebo nápovědu o různých
QGIS nástrojech.

Právě QGIS vývojáři potřebují dokumentaci pro výtváření zautomatizovaných skriptů nebo QGIS zásuvných modulů (pluginů).
Skripty a větší část pluginů jsou vytvářena v programovacím jazyku Python (viz \ref{section-python})
a menší část v programovacím jazyku C++. Instalaci pluginu lze provést z nabídky uvnitř systému QGIS
anebo pomocí ZIP souboru. Pro zobrazení pluginu v QGIS nabídce je potřeba nechat plugin
schválit správci systému QGIS, jestli splňuje všechny povinné aspekty QGIS pluginů.

Součástí systému QGIS je i Model Designer (nebo také Graphical modeler, česky Grafický modelář).
Pomocí Grafického modeláře lze tvořit modely QGIS nástrojů, což jsou uspořádané nástroje do jednoho řetězového postupu.
Pomáhá lépe se v tomto postupu orientovat, měnit vstupy funkcí, pojmenovávat nástroje a jejich parametry,
přidávat komentáře či jednotlivé nástroje seskupovat do skupin. 

\section{Python}
\label{section-python}
Python je moderní, dynamický, skriptovací programovací jazyk s veřejným zdrojovým kódem. 
Je snadný na naučení, lehko čitelný a v současné době velmi populární. Tento programovací jazyk využívají
softwaroví inženýři, matematici, datoví analytici, vědci, účetní a síťoví inženýři.
%% ML: rekl bych od strednich skol vys...  MK: našel jsem, že se to učili i děti na základních školách
Díky jednoduchosti programovacího jazyka Python se využívá k učivu na základních školách.

\begin{figure}[H] \centering
    \includegraphics[width=260pt]{./pictures/python-logo.png}
    \caption[Logo Pythonu]{Logo Pythonu \cite{python}}
	\label{fig:python-logo}                                
\end{figure} 

Důvodů, proč je tento programovací jazyk tak populární, je mnoho. Python je vyšší programovací jazyk,
což znamená lepší srozumitelnost než nižších programovacích jazyků a programy zapsané
ve vyšších jazycích jsou obvykle kratší a lépe čitelné.  Další důvodem je multiplatformnost,
což znamená, že se můžou aplikace programovacího jazyka Python sestavit a běžet na různých platformách jako 
Windows, Mac či Linux. Komunita Pythonu je obrovská, což je další výhodou tohoto jazyka.
Například jenom v Čechách je mnoho skupin na různých sociálních sítích.
Python má také spoustu knihoven, frameworků a nástrojů, které lidem usnadňují programování.

Podporuje různá programovací paradigmata
jako například objektově orientované, imperativní, procedurální nebo funkcionální.
Programovacím paradigmatem se rozumí základní programovací styl, který se liší v pojmech a abstrakcích,
které tvoří jednotlivé prvky programu, a krocích, ze kterých se výpočet skládá. \cite{wikipedia-paradigma} 
 
\section{PyQt}
Aby aplikaci vytvořenou v Pythonu mohl ovládat člověk, který se neorientuje v programování a ani nechce,
%% ML: krome GUI existuje CLI (command line interface), pote uzivatel spousti/ovlada aplikaci z prikazove radky
je nutné ke kódu aplikace vytvořit grafické uživatelské rozhraní (\zk{GUI}). Python má širokou škálu knihoven
pro vytváření \zk{GUI} jako například Tkinter, wxPython, PySide2 nebo PyQt. Právě PyQt je již standardně 
vestavěný v systému QGIS, tak se zdá jako nejlepší volbou pro tvorbu pluginů.  

PyQt je vazba Pythonu pro aplikační framework Qt vyvinutý společností RiverBank Computing Ltd.
Je k dispozici ve třech verzích: PyQt6 podporující Qt v6, PyQt5 podporující Qt v5 a PyQt4 podporující Qt v4,
avšak PyQt4 s Qt v4 již není v nových verzích systému QGIS podporována. PyQt je multiplatformní a lze nainstalovat na Windows,
macOS, Linux, iOS a Android. \cite{pyqt}

% \begin{figure}[H] \centering
%     \includegraphics[width=400pt]{./pictures/pyqt.png}
%     \caption[Schéma PyQt Pythonu]{Schéma PyQt}
% 	\label{fig:pyqt}                                
% \end{figure} 

\section{Ostatní}

Pro verzování kódu anebo pro jeho psaní byly použity i další technologie. Diplomová práce by šla tvořit i bez nich,
ale s nimi se zrychloval a usnadňoval její postup.

\subsection{GitHub}
\label{section-github}

GitHub je služba poskytující internetový hosting pro správu verzí pomocí Git, což je software pro sledování 
změn v libovolné sadě souborů. Tato služba například poskytuje řízení přístupu, sledování chyb, 
správu úkolů nebo nepřetržitou integraci. Základní služby poskytuje zdarma, ale za pokročilé funkce je nutné zaplatit.

\begin{figure}[H] \centering
    \includegraphics[width=100pt]{./pictures/github.png}
    \caption[Logo GitHubu]{Logo GitHubu \cite{github}}
	\label{fig:github}                                
\end{figure} 

\subsection{PyCharm}

PyCharm je vývojové prostředí vyvinuté firmou JetBrains s.r.o. pro programovací jazyk Python.
Usnadňuje práci s kódem, zvyšuje produktivitu a zpřehledňuje kód například zvýrazněním syntaxe Pythonu.
PyCharm obsahuje širokou kolekci plu\-ginů, které lze doinstalovat a jsou vyvíjeny uživateli.
Skrz PyCharm lze propojit i Git, takže psaný kód lze snadno zálohovat. Stejně jako GitHub PyCharm poskytuje 
základní funkce zdarma, ale profesionální verzi s pokročilejšími funkcemi je nutné si zaplatit.

\begin{figure}[H] \centering
    \includegraphics[width=100pt]{./pictures/pycharm.png}
    \caption[Logo PyCharmu]{Logo PyCharmu \cite{pycharm}}
	\label{fig:pycharm}                                
\end{figure} 
 