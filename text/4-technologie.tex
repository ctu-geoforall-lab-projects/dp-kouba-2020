\chapter{Technologie tvorby}
\label{4-technologie}

\section{QGIS}

QGIS je profesionální geografický informační systém.
Software je zdarma ke stáhnutí a je tzv. open-source (zdrojový kód je veřejně).
Zdrojový kód je publikován na GitHubu \footnote{webová služba podporující vývoj softwaru za pomoci verzovacího nástroje Git}
a vývojář softwaru může být kdokoliv, avšak potvrzovat a kontrolovat změny můžou jen ověření
přispěvatelé. Software byl vyvíjen v programovacím jazyku C++ a Python (viz \ref{section-python})

\begin{figure}[H] \centering
    \includegraphics[width=64pt]{./pictures/qgis-logo.png}
    \caption[Logo QGISu]{Logo QGISu \cite{qgis}}
	\label{fig:qgis-logo}                                
\end{figure}

QGIS je oficiálním projektem nadace OSGeo (Open Source Geospatial Foundation), což je nevládní 
nezisková organizace, jejíž cílem je podporovat a prosazovat společný vývoj otevřených geoinformačních
technologií. Běží na operačních systémech Linux, Unix, Mac OSX, Windows a Android a podporuje
řadu vektorových, rastrových a databázových formátů.

Kromě desktop verze QGISu existuje i QGIS Server, který umožňuje publikovat projekty a vrstvy
QGIS jako služby WMS, WMTS, WFS a WCS kompatibilní s OGC (Open Geospatial Consortium - mezinárodní standardizační organizace
o geoprostorových datech a službách). Dále existuje webový klient, kde lze publikovat
QGIS projekty a využívat funkce symboliky a značení. Pro mobilní zařízení tu jsou
různé aplikace jako QField, Input a IntraMaps Roam, se kterými také lze používat QGIS projekty a to 
například přímo v terénu. 

%rozhraní QGIS?

QGIS má jako většina softwarů svou dokumentaci, která je sestavena pomocí nástroje Sphinx. 
Dokumentace je taktéž open-source na GitHubu a upravovat ji může každý. Obsahuje
informace pro obyčejné desktop uživatele QGISu, pro přispěvatele dokumentace nebo pro QGIS
vývojáře.

Právě QGIS vývojáři potřebují dokumentaci pro výtváření zautomatizovaných skriptů nebo QGIS zásuvných modulů (pluginů).
Skripty a větší část pluginů jsou vytvářena v programovacím jazyku Python (viz \ref{section-python})
a menší část v programovacím jazyku C++. Instalaci pluginu lze provést z nabídky uvnitř QGISu
anebo pomocí ZIP souboru. Pro zobrazení pluginu v QGIS nabídce je potřeba nechat plugin
schválit správci QGISu, jestli splňuje všechny povinné aspekty QGIS pluginů.

\section{Python}
\label{section-python}
Python je moderní, dynamický, skriptovací programovací jazyk s veřejným zdrojovým kódem. 
Je snadný na naučení, lehko čitelný a v současné době velmi populární. Tento programovací jazyk využívají
softwaroví inženýři, matematici, datovi analytici, vědci, účetní a síťoví inženýři.
Dokonce kvůli jednoduchosti tohoto jazyka ho využívají i děti na základních školách.

Důvodů, proč je tento programovací jazyk tak populární, je mnoho. Python je vyšší programovací jazyk,
což znamená lépší srozumitelnost než nižších programovacích jazyků a programy zapsané
ve vyšších jazycích jsou obvykle kratší a lépe čitelné.  Další důvodem je multiplatformovost,
což znamená, že se můžou Pythoní aplikace sestavit a běžet na různých platformách jako 
Windows, Mac či Linux. Komunita Pythonu je obrovská, což je další výhodou tohoto jazyka.
Například jenom v Čechách je mnoho skupin na různých sociálních sítích.
Python má také spoustu knihoven, frameworků a nástrojů, které lidem usnadňují programování.

Podporuje různá programovací paradigmata
\footnote{základní programovací styl, který se liší v pojmech a abstrakcích,
které tvoří jednotlivé prvky programu, a krocích, ze kterých se výpočet skládá  
- objektově orientované, imperativní, procedurální nebo funkcionální \cite{wikipedia-paradigma}} 
jako například objektově orientované, imperativní, procedurální nebo funkcionální.

\begin{figure}[H] \centering
    \includegraphics[width=260pt]{./pictures/python-logo.png}
    \caption[Logo Pythonu]{Logo Pythonu \cite{python}}
	\label{fig:python-logo}                                
\end{figure} 

\section{PyQt}
Aby aplikaci vytvořenou v Pythonu mohl ovládat člověk, který se neorientuje v programování a ani nechce,
musí se i ke kódu aplikace vytvořit grafické uživatelské rozhraní (GUI). Python má širokou škálu knihoven
pro vytváření GUI jako například Tkinter, wxPython, PySide2 nebo PyQt. Právě PyQt je již standardně 
vestavěný v QGISu, tak se zdá jako nejlepší volbou pro tvorbu pluginů.  

PyQt je vazba Pythonu pro aplikační framework Qt vyvinutý společností RiverBank Computing Ltd.
Je k dispozici ve třech verzích: PyQt6 podporuje Qt v6, PyQt5 podporuje Qt v5 a PyQt4 podporuje Qt v4,
avšak PyQt4 s Qt v4 již není podporována. PyQt je multiplatformní a lze nainstalovat na Windows,
macOS, Linux, iOS a Android. \cite{pyqt}

\begin{figure}[H] \centering
    \includegraphics[width=400pt]{./pictures/pyqt.png}
    \caption[Schéma PyQt Pythonu]{Schéma PyQt}
	\label{fig:pyqt}                                
\end{figure} 