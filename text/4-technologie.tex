\chapter{Technologie tvorby}
\label{4-technologie}

\section{QGIS}

\begin{figure}[H] \centering
    \includegraphics[width=64pt]{./pictures/qgis-logo.png}
    \caption[Logo QGISu]{Logo QGISu \cite{qgis}}
	\label{fig:qgis-logo}                                
\end{figure}

\section{Python}
Python je moderní, dynamický, skriptovací programovací jazyk s veřejným zdrojovým kódem (tzv. open-source). 
Je snadný na naučení,lehko čitelný a v současné době velmi populární. Tento programovací jazyk využívají
softwaroví inženýři, matematici, datovi analytici, vědci, účetní a síťoví inženýři.
Dokonce kvůli jednoduchosti tohoto jazyka ho využívají i děti na základních školách.

Důvodů, proč je tento programovací jazyk tak populární, je mnoho. Python je vyšší programovací jazyk,
což znamená lépší srozumitelnost než nižších programovacích jazyků a programy zapsané
ve vyšších jazycích jsou obvykle kratší a lépe čitelné.  Další důvodem je multiplatformnost,
což znamená, že se můžou Pythoní aplikace sestavit a běžet na různých platformách jako 
Windows, Mac či Linux. Komunita Pythonu je obrovská, což je další výhodou tohoto jazyka.
Například jenom v Čechách je mnoho skupin na různých sociálních sítích.
Python má také spoustu knihoven, frameworků a nástrojů, které lidem usnadňují programování.

Podporuje různá programovací paradigmata
\footnote{základní programovací styl, který se liší v pojmech a abstrakcích,
které tvoří jednotlivé prvky programu, a krocích, ze kterých se výpočet skládá  
- objektově orientované, imperativní, procedurální nebo funkcionální \cite{wikipedia-paradigma}} 
jako například objektově orientované, imperativní, procedurální nebo funkcionální.

\begin{figure}[H] \centering
    \includegraphics[width=260pt]{./pictures/python-logo.png}
    \caption[Logo Pythonu]{Logo Pythonu \cite{python}}
	\label{fig:python-logo}                                
\end{figure} 

\section{PyQt5}

\begin{figure}[H] \centering
    \includegraphics[width=400pt]{./pictures/pyqt.png}
    \caption[Schéma PyQt5 Pythonu]{Schéma PyQt5}
	\label{fig:pyqt}                                
\end{figure} 