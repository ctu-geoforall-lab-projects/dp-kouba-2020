\chapter{Rešerše}
\label{0-reserse}

Tarifní pásma Pražské integrované dopravy jsou mimo jiné publikována jako shapefile 
\footnote{formát vektorového datového úložiště Esri pro ukládání umístění,
tvaru a atributů geografických prvků \cite{shapefile}}
na portálu \textit{opendata hlavního města Prahy} \cite{opendata}. Tento shapefile
obsahuje vektorové vrstvy polygonů.

První myšlenkou, jak se přiblížit k takovým polygonům bylo vytvoření linií mezi jednotlivými zastávkami.
K tomu slouží takzvaná triangulace, která vytvoří trojúhelníky bezi body, kde uvnitř těchto trojuhelníků  
už neleží žádné body a každý trojúhelník má vždy společnou jednu hranu. 

Takové triangulace se používají v kartografii, tak v GIS, v Dálkovém průzkumu země,
počítačové grafice, při analýze vlastností a struktury materiálů, plánování pohybu robotů
nebo při modelování přírodních jevů. \cite{bayer-delaunay}

Existuje více druhů triangulací, které využívají vždy jinou metodu konstrukce
a mají rozdílný výpočetní stupeň složitosti. 

Hladová (Greedy) triangulace se snaží vytvářet trojúhelníky s nejkratšími stranami,
které nesplňují žádnou speciální geometrickou podmínku. Její realizace je jednoduchá,
avšak důsledek toho jsou často tvarově nepěkné nebo nevhodné trojúhelníky. Má velkou výpočetní
složitost \(O(n^3)\), lze optimalizovat na \(O(n^2 \log(n))\) a v kartografii se 
nepříliš používá. \cite{vanicek}

\begin{figure}[H] \centering
    \includegraphics[width=400pt]{./pictures/triangulace-greedy.png}
    \caption[Ilustrace Greedy triangulace]{Ilustrace Greedy triangulace \cite{triangulace-greedy}}
	\label{fig:triangulace-greedy}              
\end{figure}

Další triangulací je tzv. Delaunay triangulace, která je nejčastěji používaná.
Delaunay triangulací se rozumí vytváření liniových spojnic bodů mez jednotlivými body, které si jsou sobě nejblíž,
za pomocí opsaných kružnic. Delaunay triangulace má několik vlastností. Jednou z nich je například,
že uvnitř kružnice k opsané libovolnému trojúhelníku neleží žádný jiný bod.
Zároveň Delaunay triangulace je jednoznačná, pokud žádné čtyři body neleží na kružnici.
Na rozdíl od Greedy triangulace nehodnotí kritérium délky hran. Díky maximalizaci minimálních
úhlů vytváří takové trojúhelníky, které se nejvíc blíží k rovnostranným trojúhelníkům, 
což znamená, že se snaží eliminovat trojúhelníky, které jsou ostroúhlé.

Pro výtváření konstrukce Delaunay triangulace jsou k dispozici různé algoritmy: lokální prohazování, 
inkrementální konstrukce, inkrementální vkládání, rozděl a panuj nebo sweep line. \cite{bayer-delaunay}

\begin{figure}[H] \centering
    \includegraphics[width=280pt]{./pictures/triangulace-delaunay.png}
    \caption[Ilustrace Delaunay triangulace]{Ilustrace Delaunay triangulace \cite{triangulace-delaunay}}
	\label{fig:triangulace-delaunay}              
\end{figure}

Poté existuje triangulace s minimální hmotností (anglicky Minimum Weight Triangulation, zkráceně MWT),
která má minimální celkovou délku hran. MWT je nejlepší triangulace pro interpolaci trojrozměrných oblastí,
tudíž se pro tvorbu 2D tarifních pásem úplně nehodí.

Pro vytváření hranic polygonů je konvexní obálka jednou z možností, co použít. Konvexní obálka (anglicky Convex hull)
je nejmenší konvexní mnoužinou, pokud spojnice  libovolných dvou prvků leží zcela uvnitř této množiny.
Je to jedna z nejpoužívanějších geometrických struktura, která je většinou používaná jako první odhad tvaru
prostorového tvaru.

Konstrukce konvexní obálky se nejčastěji provádějí metodami: Jarvis Scan, Graham Scan, Quick Hull nebo
Divide and Conquer. Konvexní obálku lze vytvářet i pomocí Delaunay triangulace, kdy se po dokončení
spojí všechny trojúhelníky.

\begin{figure}[H] \centering
    \includegraphics[width=400pt]{./pictures/convexHull.png}
    \caption[Konvexní obálka]{Konvexní obálka}
	\label{fig:convexHull}              
\end{figure}
 
Oproti konvexní obálce je konkávní obálka pro tvorbu složitější. Není znám přesný postup,
jak tento problém řešit, ale existují různé aproximace, které se k tomu chtějí přiblížit.
První aproximace je pomocí Delaunay triangulace, kde se vytvoří trojúhelníky, vznikne
konvexní obálka a pak se odebírají trojúhelníky s nevhodnými vlasnostmi. Typicky mají příliš ostrý nebo tupý
úhel, hraniční hodnota si můžete zvolit.
Předpokladem, aby úspěšně fungovala, je, že množina bodů má konstantní
prostorovou hustotu bodů.

Další aproximací je pomocí alpha-shapes, kde se volí parametr alpha. Představme si obrovskou masu zmrzliny,
která tvoří prostor \(R^3\) a obsahující body jako „tvrdé“ kousky čokolády. Pomocí jedné z těchto 
zmrzlinových lžiček ve tvaru koule vyřezáme všechny části zmrzlinového bloku, na které se dostaneme,
aniž bychom narazili na kousky čokolády, a tím dokonce vydlabeme otvory uvnitř 
(např. části nedosažitelné pouhým pohybem lžíce z venku). Nakonec skončíme s 
(ne nutně konvexním) objektem ohraničeným oblouky a body. Pokud nyní narovnáme všechny 
„kulaté“ plochy na trojúhelníky a úsečky, máme intuitivní popis toho, co se nazývá alpha-shape. \cite{alpha-shapes}

Další strategií je kompromisni definice mezi alpha-shapes a konkávní obálkou, tzv. alpha-concave
hull. 

\begin{figure}[H] \centering
    \includegraphics[width=296pt]{./pictures/alphashape.png}
    \caption[Konvkávní obálka pomocí alpha-shapes]{Konvkávní obálka pomocí alpha-shapes \cite{alpha-shapes-picture}}
	\label{fig:alpha-shapes-picture}              
\end{figure} 

Tarifní pásma zasahují tam, kde nejsou žádné zastávky. Pro takové vyplnění plochy
mezi zastávkami a pro místa, kde zastávky nejsou (většinou na krajích zón) není triangulace, konvexní nebo konkávní 
obálka úplně vhodná. Pro takový problém bylo další myšlenkou použít teselaci, což je vyplnění roviny pomocí jednoho
nebo více geometrických útvarů vzájemně spojených, bez překrývú a mezer. Triangulace, konvexní nebo konkávní 
obálka se mohou hodit při dalších krocích výpočtu.

Nejpoužívanější teselací v oblastí GIS je Voronoi diagram, někdy nazývána Voroného teselace, Voroného dekompozice,
Thiessenovy polygony nebo Dirichletova teselace, což je způsob rozkladu 
metrického prostoru určený vzdálenostmi k dané nespojité množině bodů v prostoru.
V našem případě se bude řešit Voronoi diagram ve 2D prostoru, tedy v rovině.

Takže na vstupu Voronoi diagramu je nějaká množina bodů a výstupem je Voronoi diagram, 
což představuje takovou množinu buněk, pro které bude platit, že každý bod
\textit{q} nálěžící množině \textit{V(p\textsubscript{i})} je blíže k bodu
\textit{p\textsubscript{i}} než k jakémukoliv
bodu \textit{p\textsubscript{j}} náležící množině \textit{P}.  \cite{bayer-voronoi}

Pro lepší chápání Voronoi diagramů je potřeba si vysvětlit její terminologii.
Vstupní množinu bodů nazýváme generátory, každý bod generuje jednu Voronoi buňku. V 
terminologii GIS se hovoří o tzv. Voronoi polygony. Tyto Voronoi buňky jsou tvořeny hranami,
které spojují dva Voronoi vrcholy. Dohromady tyto Voronoi buňky tvoří Voronoi diagram.  

\begin{figure}[H] \centering
    \includegraphics[width=400pt]{./pictures/bayer-voronoi-terminologie.png}
    \caption[Terminologie Voronoi diagramu]{Terminologie Voronoi diagramu \cite{bayer-voronoi}}
	\label{fig:bayer-voronoi-terminologie}              
\end{figure}

Voronoi diagramy mají své vlastnosti, které jsou důležité pro jejich tvorbu. V následujících
odrážkách jsou doslovně citována jejih znění z prezentace pana doc. Ing. Tomáše Bayera, Ph.D.

\begin{itemize}
\item Voronoi diagram \textit{V(P)} je planárním grafem.
\item Vrchol q Voronoi buňky \textit{\{V(p\textsubscript{i})} je průnikem 3 hran, právě když je \textit{V(P)} nedegenerovaný.
\item Pokud \textit{p\textsubscript{i}} náležící \textit{H(P)}, pak je \textit{V(p\textsubscript{i})}  otevřený. 
\item Pro každý bod \textit{p\textsubscript{i} náležící P} je \textit{V(P)} konvexní. 
\item Bod \textit{p\textsubscript{i}} je nejbližším bodem bodu \textit{p}
jestliže \textit{p} náleží \textit{\{V(p\textsubscript{i})}.
\item Každá strana \textit{q\textsubscript{i}q\textsubscript{j}}, \(i \neq j\),
je sdílena právě dvěma sousedními buňkami \textit{V(p)}. 
\item Bod q je vrcholem \textit{V(p)}, pokud existuje kružnice \textit{k(q,r)} procházející třemi
nebo více generátory \textit{p\textsubscript{i}}, \textit{p\textsubscript{j}},
\textit{p\textsubscript{k}}, a neobsahuje žádný další bod P (spojitost s \textit{DT(P)}). 
\item Kružnici \textit{k(q,r)} označujeme jako největší prázdnou kružnici ze všech prázdných kružnic se středem v bodě \textit{q}. 
\item Průměrné množství Voronoi hran ve Voronoi polygonu nepřekročí hodnotu 6. 
\item Vztah mezi počtem bodů \textit{n}, počtem hran \textit{n\textsubscript{h}}
a počtem trojúhelníků \textit{n\textsubscript{t}} teselace \textit{V(P)}:
\[ n_h \leq 3n-6\]
\[ n_t \leq 2n−5\]
\item Voronoi diagram \textit{V(P)} představuje ortografickou projekci stěn
mnohostěnu tvořeného průsečnicemi všech polorovin \textit{A\textsubscript{i}} do roviny \textit{xy}. 
\item Nechť bod \textit{p\textsubscript{i}\textsuperscript{*}} i představuje
ortografický průmět bodu \textit{p\textsubscript{i}} na povrch paraboloidu daného rovnicí:
\[ z = x^2 + y^2 \]
   
\end{itemize}

\begin{figure}[H] \centering
    \includegraphics[width=400pt]{./pictures/voronoi.png}
    \caption[Voroného polygony]{Voroného polygony}
	\label{fig:voronoi}              
\end{figure}   

Voronoi diagram má jistý vztah s Delaunay triangulací. Hraniční vstupní body
Voronoi diagramu po spojení Delaunay triangulací tvoří konvexní obálku.
Středy kružnic opsaných trojúhelníků Delaunay triangulace představují úzlové body
Voronoi diagramu.

Pro konstrukci Voronoi diagramu existují různé metody. Konstrukce lze v praxi tvořit přímo nebo nepřímo.
Pro přímou konstrukci jsou tu metody Inkrementální konstrukce, Sweep line algoritmus a
Rozděl a panuj. Nepřímá konstrukce je vytvářena přes Delaunay triangulaci skrz spojení středů
kružnic opsaných trojúhelníků a bývá nejpoužívanější. 

Využití Voronoi diagramů v oblasti GIS je široké. Například tzv. poštovní problém
pomáhá při návrhu nových supermarkétů, stanic MHD či polohy nových nemocnic.
Další využití jsou například nalezení všech sousedů, převod bodových údajů na plošné
nebo klasifikace dat. \cite{bayer-voronoi}

Pro přiblížení se výsledku k graficky estetičtější verzi je potřeba hranice polygonů
generalizovat a následně zjemnit. K tomu existuje mnoho algoritmů, ale bude zde uvedeno
jen několik vhodných a nejpoužívanějších.

Prvním algoritmem je generalizační algoritmus „Douglas-Peucker“. Algoritmus rekurzivně rozděluje čáru.
Zpočátku jsou uvedeny všechny body mezi prvním a posledním bodem. Automaticky označí první a poslední bod,
který se má zachovat. Poté najde bod, který je nejvzdálenější od úsečky s prvním a posledním bodem 
jako koncovými body. Tento bod je zjevně nejvzdálenější na křivce od přibližné úsečky mezi koncovými body. 
Pokud je bod blíže než parametr epsilon, který je volitelný, k úsečce, mohou být všechny body, které nejsou aktuálně označeny jako zachováné, 
zahozeny, aniž by zjednodušená křivka byla horší než parametr epsilon. Pokud je nejvzdálenější bod od úsečky větší než parametr epsilon od aproximace,
musí být tento bod zachován. Algoritmus se rekurzivně volá sebe samotného s prvním bodem a nejvzdálenějším bodem 
a poté nejvzdálenějším bodem a posledním bodem, který zahrnuje nejvzdálenější bod označený jako ponechaný. 
Tento algoritmus je jeden z nejlepších generalizacních algoritmů a je velmi často
implementován v GIS software. \cite{bayer-douglas}

\begin{figure}[H] \centering
    \includegraphics[width=400pt]{./pictures/douglas.png}
    \caption[Znázornění algoritmu „Douglas-Peucker“]{Znázornění algoritmu „Douglas-Peucker“ \cite{bayer-douglas}}
	\label{fig:douglas}              
\end{figure} 

Dalším algoritmem je zjemňovací/vyhlazovací algoritmus Polynomická aproximace metodou Exponenciálního jádra
(anglicky Polynomial Approximation with Exponential Kernel, zkráceně PAEK).
Tento algoritmus vyvinula společnost Esri pro nástroj Smooth Line a je jejím soukromým vlastnictvím.
Ačkoliv se zdá být ideální, tak není použitelný pro jiné platformy kvůli nezveřejněnému algoritmu.

Podobným algoritmem jako PAEK je algoritmus Chaikinův. 
Chaikin (autor algoritmu, po kterém je i algoritmus pojmenován) využil fixních poměrů na odříznutí rohů, takže byly všechny rozřezány stejně. 
Při matematickém zápisu postupuje Chaikinova metoda následovně: Dostaneme kontrolní polygon 
\textit{\{P\textsubscript{0}, P\textsubscript{1}, ..., P\textsubscript{n}\}},
tento kontrolní polygon vylepšíme vygenerováním nové posloupnosti řídicích bodů 
\[ \{Q_0, R_0, Q_1, R_1, ...,  Q_{n−1}, R_{n−1}\} \]                                    
kde každá nová dvojice bodů Q\textsubscript{i}, R\textsubscript{i} je třeba brát v poměru \(\frac{1}{4}\)
a \(\frac{3}{4}\) mezi koncovými body segmentu čáry \(\overline{P\textsubscript{i}P\textsubscript{i+1}}\).
\[Q_i = \frac{3}{4}P_i + \frac{1}{4}P_{i+1}\]
\[R_i = \frac{1}{4}P_i + \frac{3}{4}P_{i+1}\]
Tyto 2 nové body lze považovat za nový řídicí polygon - vylepšení původního řídicího polygonu. 
Tento postup lze pro tentýž úsek opakovat, čímž vznikne vyhlazená čára. \cite{chaikin} 

\begin{figure}[H] \centering
    \includegraphics[width=400pt]{./pictures/chaiken.png}
    \caption[Znázornění Chaikinova algoritmu]{Znázornění Chaikinova algoritmu \cite{bayer-douglas}}
	\label{fig:chaiken}              
\end{figure} 