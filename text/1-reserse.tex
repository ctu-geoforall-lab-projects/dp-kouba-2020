\chapter{Rešerše}
\label{0-reserse}

Tarifní pásma Pražské integrované dopravy jsou mimo jiné publikována jako shapefile 
\footnote{formát vektorového datového úložiště Esri pro ukládání umístění,
tvaru a atributů geografických prvků \cite{shapefile}}
na portálu \textit{opendata hlavního města Prahy} \cite{opendata}. Tento shapefile
obsahuje vektorové vrstvy polygonů.

První myšlenkou, jak se přiblížit k takovým polygonům, byl algoritmus Delaunay triangulace.
Tento algoritmus vytváří liniové spojnice bodů mez jednotlivými body, které si jsou sobě nejblíž,
za pomocí opsaných kružnic. Delaunay triangulace má několik vlastností. Jednou z nich je například,
že uvnitř kružnice k opsané libovolnému trojúhelníku neleží žádný jiný bod.
Zároveň Delaunay triangulace je jednoznačná, pokud žádné čtyři body neleží na kružnici. \cite{bayer-delaunay}

Další nápadem byla aplikace Voronoi diagram algoritmu, což je způsob dekompozice 
metrického prostoru určený vzdálenostmi k dané diskrétní množině bodů v prostoru.
Tento algoritmus má taky několik vlastností, jež budou prezentovány v dalších kapitolách 
(viz \ref{voronoi-vlastnosti}). 