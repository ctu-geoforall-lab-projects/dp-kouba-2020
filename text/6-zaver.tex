\chapter*{Závěr}
\addcontentsline{toc}{chapter}{Závěr}
\markboth{ZÁVĚR}{}
\label{6-zaver}

%% ML: odstavec ~ dlouhe souveti, opakovani spojky "a"
Primárním cílem této diplomové práce bylo rozšířit plugin systému QGIS 
o automatické vytváření tarifních pásem Pražské integrované dopravy za pomocí GTFS datasetu,
což byly podkladová data zveřejňovaná organizací ROPID. Mezikrokem k primárnímu cíly
bylo zveřejnění pluginu do repozitáře QGIS. Sekundární cíl bylo vytvořit schématický plán
nesplňující pravidla 

% Cílem této diplomové práce bylo za pomocí GTFS datasetu rozšířit plugin systému QGIS pro zpracování dat 
% ve formátu GTFS a najít a vhodně implementovat způsob vizualizace tarifních pásem Pražské integrované dopravy 
% (PID) generovaných automaticky z podkladových dat.

Mezikrok nutný pro pokračování vývoje pluginu byl splněn ve formě aktualizace na verzi pluginu 1.0.
V této aktualizaci byly opraveny různé chyby, doplněno uživatelské rozhraní a proces pluginu uložen do pozadí. 
Následné byl publikován v oficiálním repozitáři systému QGIS s názvem pluginu \textit{GTFS Loader}. 
K 12.5.2021 bylo v záložce \textit{Plugins} uvedeno 412 unikátních stáhnutí pluginu.

%% ML: tvorby a vizualizace
Implementace vizualizace tarifních pásem byla vytvářena jako verze pluginu 2.0. Výsledek
%% ML: pojem "obarveni" nezni prilis technicky
byl tvořen jako volitelný výstup pluginu ve formě vektorové vrstvy s obarvením jednotlivých tarifních pásem.
%% ML: zkuste najit vhodnejsi termin nez "tvar" (napr. prubeh tarifnich pasem a pod)
Výsledný tvar byl podobný oficiální verzi publikované veřejně organizací ROPID, ale byl doplněn o tzv. hraniční pásma,
která pokrývala větší seskupení hraničních zastávek. Pro ostatní nepokryté hraniční zastávky 
se nepodařil najít způsob, jak je napřímo propojit s hranicemi tarifních pásem.

Jak již bylo zmíněno, proces pluginu pro vytváření tarifních pásem nezvládá správně zpracovat hraniční zastávky.
Zároveň pro oblasti, ve kterých jsou umístěny zastávky velmi vzácně, vzníkají "díry" mezi tarifními pásmy.
Tyto "díry" se ve snaze pro lepší tvar eliminovaly, ale kvůli tomu vznikl další problém překrývajících se tarifních pásem.
Každopádně pro lepší tvar je potřeba manuální editace vektorové vrstvy.

%% ML: prvni vetu preformulejte
Také bylo dáno pracovníky organizace ROPID pravidlo o "nerozřezávání" obcí na hranicích tarifních pásem,
což znamená vyhýbání se hranic tarifních pásem větší zastavěné oblasti.
%% ML: neuvadel bych to jako duvod, proc jste od toho upustil. Nebylo to prani ze strany ROPIDu?
Po zjištění, že oficiální veřejná tarifní pásma publikována stejnojmennou organizací toto také nesplňují,
bylo z tohoto pravidla upuštěno. Současně pro realizaci tohoto pásma by byla potřeba načítat vektorové vrstvy
%% ML: dotcenych obci
obcí ČR, což by razantně zvýšilo datovou náročnost pluginu.

%% ML: Jaky postup? Specifikujte presneni (Postup navrzeny v ramci DP, a pod)...
Tento postup není rozhodně ideální a je na něm co vylepšovat. Předmět úpravy a vylepšení pluginu
%% ML: vynechte expresivni vyrazy typu "tak ci tak"
%% ML: pridejte odkaz na repozitar (a konretni vetev)
se může stát dalším tématem semestrálních či závěrečných prací. Tak či tak zdrojový kód
je již nyní veřejně přístupný v repozitáři pluginu a může k němu kdokoliv přidávat
návrhy na vylepšení.  

