\chapter*{Závěr}
\addcontentsline{toc}{chapter}{Závěr}
\markboth{ZÁVĚR}{}
\label{6-zaver}

Cílem této diplomové práce bylo rozšířit plugin systému QGIS pro zpracování dat 
ve formátu GTFS a najít a vhodně implementovat způsob vizualizace tarifních pásem Pražské integrované dopravy 
(PID) generovaných automaticky z podkladových dat.

První cíl byl splněn ve formě aktualizace pluginu na verzi 1.0 a následné její publikace v oficiálním
repozitáři systému QGIS. K 8.5.2021 bylo v záložce Plugins uvedeno 400 unikátních stáhnutí pluginu.

Implementace vizualizace tarifních pásem byla vytvářena jako verze pluginu 2.0. Výsledek
byl tvořen jako volitelný výstup pluginu ve formě vektorové vrstvy s obarvením jednotlivých tarifních pásem.
Výsledný tvar byl podobný oficiální verzi publikované veřejně organizací ROPID, ale byl doplněn o tzv. hraniční pásma,
která pokrývala větší seskupení hraničních zastávek. Pro ostatní nepokryté hraniční zastávky 
se nepodařil najít způsob, jak je napřímo propojit s hranicemi tarifních pásem.

Jak již bylo zmíněno, proces pluginu pro vytváření tarifních pásem nezvládá správně zpracovat hraniční zastávky.
Zároveň pro oblasti, ve kterých jsou umístěny zastávky velmi vzácně, vzníkají "díry" mezi tarifními pásmy.
Tyto "díry" se ve snaze pro lepší tvar eliminovaly, ale kvůli tomu vznikl další problém překrývajících se tarifních pásem.
Každopádně pro lepší tvar je potřeba manuální editace vektorové vrstvy.

Také bylo dáno pracovníky organizace ROPID pravidlo o "nerozřezávání" obcí na hranicích tarifních pásem,
což znamená vyhýbání se hranic tarifních pásem větší zastavěné oblasti.
Po zjištění, že oficiální veřejná tarifní pásma publikována stejnojmennou organizací toto také nesplňují,
bylo z tohoto pravidla upuštěno. Současně pro realizaci tohoto pásma by byla potřeba načítat vektorové vrstvy
obcí ČR, což by razantně zvýšilo datovou náročnost pluginu.

Tento postup není rozhodně ideální a je na něm co vylepšovat. Předmět úpravy a vylepšení pluginu
se může stát dalším tématem semestrálních či závěrečných prací. Tak či tak zdrojový kód
je již nyní veřejně přístupný v repozitáři pluginu a může k němu kdokoliv přidávat
návrhy na vylepšení.  

