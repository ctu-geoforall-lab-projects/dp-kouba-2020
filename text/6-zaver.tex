\chapter*{Závěr}
\addcontentsline{toc}{chapter}{Závěr}
\markboth{ZÁVĚR}{}
\label{6-zaver}

Cílem této diplomové práce bylo rozšířit plugin systému QGIS pro zpracování dat 
ve formátu GTFS a najít a vhodně implementovat způsob vizualizace tarifních pásem Pražské integrované dopravy 
(PID) generovaných automaticky z podkladových dat.

První cíl byl splněn ve formě aktualizace pluginu na verzi 1.0 a následné její publikace v oficiálním
repozitáři systému QGIS. K 8.5.2021 bylo v záložce Plugins uvedeno 398 unikátních stáhnutí pluginu.

Implementace vizualizace tarifních pásem byla tvořena jako verze pluginu 2.0. Výsledek
byl tvořen jako volitelný výstup pluginu ve formě vektorové vrstvy s obarvením jednotlivých tarifních pásem.
Výsledný tvar byl podobný oficiální verzi publikované veřejně organizací ROPID, ale byl doplněn o tzv. hraniční pásma,
která pokrývala větší seskupení hraničních zastávek. Pro ostatní nepokryté hraniční zastávky 
se nepodařil najít způsob, jak je napřímo propojit s hranicemi tarifních pásem. 

