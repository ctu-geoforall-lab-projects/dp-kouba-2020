\chapter*{Závěr}
\addcontentsline{toc}{chapter}{Závěr}
\markboth{ZÁVĚR}{}
\label{6-zaver}

Primárním cílem této diplomové práce bylo rozšířit plugin systému QGIS 
o automa\-tické vytváření exaktního tvaru tarifních pásem Pražské integrované dopravy, 
jehož referenční podoba je publikována veřejně organizací ROPID.  
Podkladovými daty byl GTFS dataset taktéž zveřejňovaný organizací ROPID.
Mezikrokem k primárnímu cíly bylo zveřejnění pluginu do repozitáře QGIS. Sekundárním cílem bylo vytvořit schématický model pro 
informativní účely, který by se tolik neblížil k referenční podobě tarifních pásem.

Mezikrok nutný pro pokračování vývoje byl splněn ve formě aktualizace na verzi pluginu 1.0.
V této aktualizaci byly opraveny některé chyby, doplněno uživatelské rozhraní a proces pluginu přeložen na pozadí. 
Následné byl publikován v oficiálním repozitáři systému QGIS s názvem \textit{GTFS Loader}. 
K 12.5.2021 bylo v záložce \textit{Plugins} uvedeno 412 unikátních stáhnutí pluginu.

Implementace tvorby tarifních pásem byla vytvářena jako experimentální verze pluginu 2.0. Výsledek
byl tvořen jako volitelný výstup pluginu ve formě vektorové vrstvy se symbologií jednotlivých tarifních pásem.
Výsledný průběh tarifních pásem byl podobný oficiální verzi publikované veřejně organizací ROPID, ale byl doplněn o tzv. hraniční tarifní pásma,
která pokrývala větší seskupení hraničních zastávek. Pro ostatní nepokryté hraniční zastávky 
se nepodařil najít způsob, jak je napřímo propojit s hranicemi tarifních pásem.

Jak již bylo zmíněno, proces pluginu pro vytváření tarifních pásem nezvládá správně zpracovat hraniční zastávky.
Zároveň pro oblasti, ve kterých jsou umístěny zastávky velmi vzácně, vznikají \uv{díry} mezi tarifními pásmy.
Tyto \uv{díry} se ve snaze pro lepší tvar eliminovaly, ale kvůli tomu vznikl další problém překrývajících se tarifních pásem.
Každopádně pro lepší tvar je potřeba manuální editace vektorové vrstvy.

Zaměstnanci organizace ROPID bylo dáno pravidlo o nekřížení se intravilánu obce s hranicemi tarifních pásem.
Toto pravidlo mělo mít za následek vyhýbání se hranic tarifních pásem s většími zastavěnými oblastmi.
Později bylo kvůli obtížnosti s uskutečněním tohoto pravidla upuštěno na přání ze strany pracovníků z organizace ROPID.
Současně pro realizaci tohoto pravidla by bylo potřeba načítat vektorové vrstvy
dotčených obcí ČR, což by razantně zvýšilo datovou náročnost pluginu.

Postup navržený v rámci diplomové práce není rozhodně ideální a je na něm co vylepšovat. Předmět úpravy a vylepšení pluginu
se může stát dalším tématem semestrálních či závěrečných prací. Zdrojový kód
je již nyní veřejně přístupný v repozitáři pluginu (odkaz \href{https://github.com/ctu-geoforall-lab/qgis-gtfs-plugin/tree/pid\_zones}
{\underline{zde}}) a může k němu kdokoliv přidávat
návrhy na vylepšení.  

