\chapter*{Úvod}
\addcontentsline{toc}{chapter}{Úvod}
\markboth{ÚVOD}{}
\label{0-uvod}

Pro cestování po Praze a okolí je často nejlepší volbou použít veřejnou dopravu.
Tu zahrnuje metro, tramvaje, železnici, městské a příměstské autobusové linky,
lanovou dráhu na Petřín a také některé přívozy. Celý tento systém se nazývá 
Pražská integrovaná doprava, který je postupně integrován společnými přepravními
a tarifními podmínkami a jednotným dopravním řešením včetně koordinace jízdních řádů.

Systém Pražské integrované dopravy je rozdělen do dvanácti tarifních pásem, pro
které platí různé jízdní doklady. Rozlohově jsou po celé Praze, na většině
Středočeského kraje a dokonce i na částech Plzeňského a Ústeckého kraje, či na Vysočině.
Více o systému Pražské integrované dopravy je v kapitole \ref{3-teorie-pid}.

Tarifní pásma Pražské integrované dopravy jsou zatím modelována manuálně a nejsou
nikterak zautomatizována. Cílem této diplomové práce bude vytvořit postup pro
automatické vyhotovení exaktního modelu tarifních pásem a co nejvíce se přiblížit k jejich
oficiální podobě vydávaným organizace ROPID. Alternativním cílem je vytvořit schématický model, který nesplňuje tolik
pravidel pro tvorbu tarifních pásem, ale má informační hodnotu pro mapy menších měřítek
například pro různé plánky Pražské integrované dopravy. V postupu se bude využívat GTFS 
dataset, o kterém se mimo jiné píše v kapitole \ref{2-teorie-gtfs}.

Tato závěrečná práce navazuje na vývoj zásuvného modulu
v softwaru QGIS, u kterého byla započata tvorba v předmětu Free software GIS
v 5. semestru magisterského studia s mými spolužáky Adrianou Brezničanovou a Jaroslavem
Zemanem. O softwaru QGIS a dalších technologiích tvorby se pojednává v kapitole 
\ref{4-technologie}.

V poslední kapitole \ref{5-postup} je popsán postup tvorby rozšíření zásuvného modulu.
Zahrnuje i několik slepých uliček postupu, do kterých se muselo vydat kvůli zjištění
vhodnějšího postupu.    



