\chapter*{Úvod}
\addcontentsline{toc}{chapter}{Úvod}
\markboth{ÚVOD}{}
\label{0-uvod}

Pro cestování po Praze a okolí je často nejlepší volbou použít veřejnou dopravu. 
Tu zahrnuje metro, tramvaje, železnici, městské a příměstské autobusové linky.
Kromě těchto dopravních prostředků je do veřejné dopravy zahrnuta lanová dráha na Petřín a také některé přívozy. 
Celý tento systém se nazývá Pražská integrovaná doprava, který je postupně integrován společnými přepravními
a tarifními podmínkami a~jednotným dopravním řešením včetně koordinace jízdních řádů.

Systém Pražské integrované dopravy je rozdělen do dvanácti tarifních pásem, pro
které platí různé jízdní doklady. Rozlohově jsou po celé Praze, na většině
Středočeského kraje a dokonce i na částech Plzeňského a Ústeckého kraje, či na Vysočině.
Více o systému Pražské integrované dopravy je uvedeno v kapitole \ref{3-teorie-pid}.

Postup tvorby tarifních pásem Pražské integrované dopravy je poloautomatizovaný
a manuální zásah tvůrce je téměř ke všemu nutný.
Primárním cílem této diplomové práce bylo za pomocí zastávek z GTFS datasetu vytvořit 
postup a jeho následnou implementaci pro automatické vyhotovení exaktního modelu tarifních pásem.
Implementace postupu se měla vkládat do předpřipraveného zásuvného mo\-dulu v~systému QGIS.
Výsledek procesu tvorby se měl co nejvíce přiblížit k oficiální podobě tarifních pásem
vydávanou organizací ROPID. 
Oficiální podoba tarifních pásem je veřejně publikována na portálu \textit{Opendata hlavního města Prahy}.
O GTFS datasetu se píše v kapitole \ref{2-teorie-gtfs}.

Nutně vyhotovitelným mezikrokem k dosažení primárního cíle bylo dokončení
vývoje zásuvného modulu do uživatelsky přívětivější verze, která mohla být nahrána do QGIS repozitáře.
Vývoj zásuvného modulu započal s mými spolužáky Adrianou Brezničanovou a Jaroslavem
Zemanem v 5. semestru magisterského studia v~předmětu Free software GIS.
O softwaru QGIS a dalších použitých technologiích se pojednává v kapitole \ref{4-technologie}. 

Sekundárním cílem bylo vytvořit schématický model, který nesplňuje tolik
pravi\-del pro tvorbu tarifních pásem, ale má informační hodnotu pro mapy menších měřítek
například pro různé plánky Pražské integrované dopravy.

V poslední kapitole \ref{5-postup} je popsán proces tvorby rozšíření zásuvného modulu.
Zahrnuje i několik nepodařených pokusů, které byly v rámci tvorby diplomové práce vyzkoušeny a následně zavrhnuty 
kvůli zjištění vhodnější verze postupu.
 
% do kterých se muselo vydat kvůli zjištění
% vhodnějšího postupu.    
% 
% 
% %% ML: slova "zatim" bych vynechal
% Tarifní pásma Pražské integrované dopravy jsou zatím modelována manuálně a nejsou
% %% ML: a jejich tvorba neni nikterak ...
% %% ML: uvod ma byt v minulem case
% nikterak zautomatizována. Cílem této diplomové práce bude vytvořit postup pro
% automatické vyhotovení exaktního modelu tarifních pásem a co nejvíce se přiblížit k jejich
% %% ML: alternativnim -> druhotnym
% oficiální podobě vydávaným organizací ROPID. Alternativním cílem je vytvořit schématický model, který nesplňuje tolik
% pravidel pro tvorbu tarifních pásem, ale má informační hodnotu pro mapy menších měřítek
% například pro různé plánky Pražské integrované dopravy. V postupu se bude využívat GTFS 
% dataset, o kterém se mimo jiné píše v kapitole \ref{2-teorie-gtfs}.
% 
% %% ML: Misto "zaverecna" je presnejsi uvadet "diplomova"
% Tato diplomová práce navazuje na vývoj zásuvného modulu
% %% ML: vetu prepiste: "u ktereho byla zapocata tvorba" nezni prilis pekne
% v softwaru QGIS, u kterého byla započata tvorba v předmětu Free software GIS
% v 5. semestru magisterského studia s mými spolužáky Adrianou Brezničanovou a Jaroslavem
% %% ML: "technologie tvorby" ?
% Zemanem. O softwaru QGIS a dalších použitých technologiích se pojednává v kapitole 
% \ref{4-technologie}.
% 
% V poslední kapitole \ref{5-postup} je popsán postup tvorby rozšíření zásuvného modulu.
% %% ML: posledni vetu prepiste "do kterych se musel vydat" zni opravdu kostrbate...
% Zahrnuje i několik slepých uliček postupu, do kterých se muselo vydat kvůli zjištění
% vhodnějšího postupu.    



                     