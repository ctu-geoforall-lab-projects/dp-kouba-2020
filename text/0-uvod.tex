\chapter*{Úvod}
\addcontentsline{toc}{chapter}{Úvod}
\markboth{ÚVOD}{}
\label{1-uvod}

Pro cestování po Praze a okolí je často nejlepší volbou použít veřejnou dopravu.
Tu zahrnuje metro, tramvaje, železnici, městské a příměstské autobusové linky,
lanovou dráhu na Petřín a také některé přívozy. Celý tento systém se nazývá 
Pražská integrovaná doprava, který je postupně integrován společnými přepravními
a tarifními podmínkami a jednotným dopravním řešením včetně koordinace jízdních řádů.

Systém Pražské integrované dopravy je rozdělen do dvanácti tarifních pásem, pro
které platí různé jízdní doklady. Rozlohově jsou po celé Praze, na většině
Středočeského kraje a dokonce i na částech Plzeňského a Ústeckého kraje, či na Vysočině.

Tarifní pásma Pražské integrované dopravy jsou zatím modelována manuálně a nejsou
nikterak zautomatizována. Cílem této diplomové práce bude vytvořit postup pro
automatické vyhotovení tarifních pásem a co nejvíce se přiblížit k jejich
oficiální podobě. Tento postup bude publikován jako pokračování vývoje zásuvného modulu
v softwaru QGIS, u kterého byla započata tvorba v předmětu Free software GIS
v 5. semestru magisterského studia.

